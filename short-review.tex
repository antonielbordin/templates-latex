% ============================================================================
% Template para Short Review de Artigos Científicos
% Com normas ABNT e formatação adequada
% ============================================================================

\documentclass[12pt,a4paper]{article}

% ========================= PACOTES ESSENCIAIS ===============================
\usepackage[utf8]{inputenc}
\usepackage[T1]{fontenc}
\usepackage[brazil]{babel}
\usepackage{times}
\usepackage{geometry}
\usepackage{setspace}
\usepackage{indentfirst}
\usepackage{titlesec}

% Pacote para citações e referências no padrão ABNT
\usepackage[alf,abnt-emphasize=bf]{abntex2cite}

% Configuração das margens
\geometry{
  a4paper,
  left=3cm,
  right=2cm,
  top=3cm,
  bottom=2cm
}

% Espaçamento entre linhas
\onehalfspacing

% Configuração dos títulos de seções
\titleformat{\section}
  {\normalfont\fontsize{12}{15}\bfseries\uppercase}{\thesection}{1em}{}
\titleformat{\subsection}
  {\normalfont\fontsize{12}{15}\bfseries}{\thesubsection}{1em}{}

% ======================= INFORMAÇÕES DO DOCUMENTO ===========================
\title{\textbf{SHORT REVIEW: [TÍTULO DO ARTIGO REVISADO]}}
\author{Seu Nome\thanks{email@instituicao.edu.br}}
\date{\today}

% ============================================================================
\begin{document}

\maketitle

% ======================= DADOS DO ARTIGO REVISADO ===========================
\section*{Informações do Artigo Revisado}

\noindent\textbf{Título:} Título Completo do Artigo\\
\textbf{Autores:} Nome dos Autores\\
\textbf{Periódico/Conferência:} Nome da Publicação, v. X, n. Y, p. Z-W, Ano.\\
\textbf{DOI/URL:} [inserir DOI ou link]\\
\textbf{Palavras-chave:} palavra1; palavra2; palavra3

% ========================== RESUMO EXECUTIVO ================================
\section*{Resumo Executivo}

Apresente aqui uma síntese concisa do artigo revisado (150-250 palavras). Inclua:
\begin{itemize}
  \item O problema ou questão de pesquisa abordada
  \item A metodologia empregada
  \item Os principais resultados
  \item As conclusões mais relevantes
\end{itemize}

% =========================== INTRODUÇÃO =====================================
\section{Introdução}

Contextualize o tema do artigo revisado, apresentando:
\begin{itemize}
  \item A relevância do tema na área de conhecimento
  \item Os objetivos do artigo original
  \item A justificativa para a realização desta revisão
\end{itemize}

Exemplo de citação direta curta: segundo \citeonline{autor2023}, "texto citado com menos de três linhas".

Exemplo de citação indireta: os estudos recentes demonstram avanços significativos na área \cite{autor2023}.

% ==================== SÍNTESE DO CONTEÚDO ===================================
\section{Síntese do Conteúdo}

\subsection{Fundamentação Teórica}

Descreva as principais teorias, conceitos e trabalhos relacionados que fundamentam o artigo revisado.

\subsection{Metodologia}

Resuma a metodologia utilizada pelos autores:
\begin{itemize}
  \item Tipo de pesquisa (qualitativa, quantitativa, mista)
  \item Instrumentos de coleta de dados
  \item Procedimentos de análise
  \item População/amostra estudada (se aplicável)
\end{itemize}

\subsection{Principais Resultados}

Apresente os resultados mais significativos encontrados no estudo, organizando-os de forma clara e objetiva.

Para citações longas (mais de três linhas), use:

\begin{citacao}
Texto da citação direta longa, com mais de três linhas. Este formato deve ser usado para preservar a integridade de passagens importantes do texto original. A citação deve ter recuo de 4cm da margem esquerda, fonte menor e espaçamento simples \cite{autor2023}.
\end{citacao}

% ======================= ANÁLISE CRÍTICA ====================================
\section{Análise Crítica}

\subsection{Pontos Fortes}

Identifique e discuta os aspectos positivos do artigo:
\begin{itemize}
  \item Originalidade da contribuição
  \item Rigor metodológico
  \item Clareza na apresentação
  \item Relevância dos resultados
  \item Qualidade das referências
\end{itemize}

\subsection{Limitações e Pontos de Melhoria}

Apresente uma análise crítica construtiva sobre:
\begin{itemize}
  \item Possíveis limitações metodológicas
  \item Gaps ou lacunas não exploradas
  \item Aspectos que poderiam ser mais aprofundados
  \item Sugestões de melhorias ou extensões
\end{itemize}

\subsection{Contribuições para a Área}

Discuta como o artigo contribui para o avanço do conhecimento na área, incluindo:
\begin{itemize}
  \item Inovações teóricas ou metodológicas
  \item Aplicações práticas
  \item Direcionamentos para pesquisas futuras
\end{itemize}

% ======================== CONSIDERAÇÕES FINAIS ==============================
\section{Considerações Finais}

Sintetize sua avaliação geral do artigo, destacando:
\begin{itemize}
    \item Avaliação geral da qualidade e relevância
    \item Recomendação de leitura (para quem seria útil)
    \item Possíveis impactos na área
\end{itemize}

% ========================= REFERÊNCIAS ======================================
% As referências são geradas automaticamente a partir do arquivo .bib
% Use o comando \cite{chave} ou \citeonline{chave} no texto

\bibliography{referencias}

% ============================================================================
% ARQUIVO referencias.bib (criar separadamente)
% ============================================================================
% Exemplo de entradas no arquivo .bib:
%
% @article{autor2023,
%   author = {Sobrenome, Nome},
%   title = {Título do Artigo},
%   journal = {Nome do Periódico},
%   year = {2023},
%   volume = {10},
%   number = {2},
%   pages = {100--120},
%   doi = {10.xxxx/xxxxx}
% }
%
% @book{autorlivro2022,
%   author = {Sobrenome, Nome},
%   title = {Título do Livro},
%   publisher = {Editora},
%   address = {Cidade},
%   year = {2022}
% }
%
% @inproceedings{autorconferencia2023,
%   author = {Sobrenome, Nome},
%   title = {Título do Trabalho},
%   booktitle = {Anais da Conferência},
%   year = {2023},
%   pages = {1--10},
%   address = {Cidade}
% }
% ============================================================================

\end{document}