% ============================================================================
% TEMPLATE DE ARTIGO CIENTÍFICO - FORMATO ABNT
% Baseado em abnTeX2 - Normas ABNT NBR 6022:2018
% ============================================================================

\documentclass[
    % Opções da classe memoir
    article,            % Indica que é um artigo acadêmico
    11pt,               % Tamanho da fonte
    oneside,            % Para impressão em apenas uma face
    a4paper,            % Tamanho do papel
    % Opções da classe abntex2
    chapter=TITLE,      % Títulos de capítulos convertidos em letras maiúsculas
    section=TITLE,      % Títulos de seções convertidos em letras maiúsculas
    %subsection=TITLE,  % Títulos de subseções convertidos em letras maiúsculas
    %subsubsection=TITLE,% Títulos de subsubseções convertidos em letras maiúsculas
    % Opções do pacote babel
    english,            % Idioma adicional para hifenização
    french,             % Idioma adicional para hifenização
    spanish,            % Idioma adicional para hifenização
    brazil,             % O último idioma é o principal do documento
]{abntex2}

% ==================== PACOTES ESSENCIAIS ====================================
\usepackage{lmodern}            % Usa a fonte Latin Modern
\usepackage[T1]{fontenc}        % Seleção de códigos de fonte
\usepackage[utf8]{inputenc}     % Codificação do documento
\usepackage{indentfirst}        % Indenta o primeiro parágrafo de cada seção
\usepackage{color}              % Controle das cores
\usepackage{graphicx}           % Inclusão de gráficos
\usepackage{microtype}          % Para melhorias de justificação
\usepackage{lipsum}             % Geração de texto dummy (remover em versão final)

% ==================== PACOTES DE CITAÇÕES ===================================
\usepackage[brazilian,hyperpageref]{backref}    % Páginas com citações na bibliografia
\usepackage[alf]{abntex2cite}   % Citações padrão ABNT

% ==================== PACOTES ADICIONAIS ====================================
\usepackage{amsmath,amssymb,amsfonts}  % Matemática avançada
\usepackage{booktabs}                   % Tabelas profissionais
\usepackage{multirow}                   % Tabelas com múltiplas linhas
\usepackage{subfig}                     % Subfiguras
\usepackage{listings}                   % Listagens de código
\usepackage{algorithm}                  % Algoritmos
\usepackage{algpseudocode}             % Pseudocódigo

% ==================== CONFIGURAÇÕES DE LISTAGENS ============================
\lstset{
    basicstyle=\footnotesize\ttfamily,
    numbers=left,
    numberstyle=\tiny,
    stepnumber=1,
    numbersep=5pt,
    backgroundcolor=\color{white},
    showspaces=false,
    showstringspaces=false,
    showtabs=false,
    frame=single,
    rulecolor=\color{black},
    tabsize=2,
    captionpos=b,
    breaklines=true,
    breakatwhitespace=false,
}

% ==================== CONFIGURAÇÕES DE APARÊNCIA ============================
\definecolor{blue}{RGB}{41,5,195}

% Configurações do pacote backref
\renewcommand{\backrefpagesname}{Citado na(s) página(s):~}
\renewcommand{\backref}{}
\renewcommand*{\backrefalt}[4]{
    \ifcase #1
        Nenhuma citação no texto.
    \or
        Citado na página #2.
    \else
        Citado #1 vezes nas páginas #2.
    \fi}

% Informações do PDF
\makeatletter
\hypersetup{
    pdftitle={\@title}, 
    pdfauthor={\@author},
    pdfsubject={Artigo Científico},
    pdfcreator={LaTeX with abnTeX2},
    pdfkeywords={palavra-chave1}{palavra-chave2}{palavra-chave3}, 
    colorlinks=true,        % false: boxed links; true: colored links
    linkcolor=blue,         % color of internal links
    citecolor=blue,         % color of links to bibliography
    filecolor=magenta,      % color of file links
    urlcolor=blue,
    bookmarksdepth=4
}
\makeatother

% Espaçamentos entre linhas e parágrafos
\setlength{\parindent}{1.3cm}
\setlength{\parskip}{0.2cm}

% ==================== INFORMAÇÕES DO ARTIGO =================================
\titulo{Título do Artigo Científico: Um Estudo sobre Metodologia de Pesquisa}

\autor{Nome Completo do Primeiro Autor\thanks{Graduando em Ciência da Computação, 
Universidade Federal de Exemplo. E-mail: autor1@email.com}
\and Nome Completo do Segundo Autor\thanks{Doutorando em Ciência da Computação, 
Universidade Federal de Exemplo. E-mail: autor2@email.com}
\and Prof. Dr. Nome do Orientador\thanks{Professor do Departamento de Computação, 
Universidade Federal de Exemplo. E-mail: orientador@email.com}}

\local{Brasil}
\data{2024}

% ============================================================================
% INÍCIO DO DOCUMENTO
% ============================================================================
\begin{document}

\selectlanguage{brazil}
\frenchspacing 

% ============================================================================
% TÍTULO E AUTORES
% ============================================================================
\maketitle

% ============================================================================
% RESUMO EM PORTUGUÊS
% ============================================================================
\begin{resumoumacoluna}
Este artigo apresenta uma visão geral sobre a elaboração de artigos científicos 
seguindo as normas da Associação Brasileira de Normas Técnicas (ABNT). O resumo 
deve conter entre 150 e 250 palavras, apresentando de forma concisa o objetivo, 
a metodologia, os principais resultados e as conclusões do trabalho. Não deve 
conter citações bibliográficas. O texto deve ser redigido em parágrafo único, 
com espaçamento simples e justificado. Recomenda-se o uso da terceira pessoa 
do singular e de verbos na voz ativa. Este template foi desenvolvido utilizando 
a classe abnTeX2, que implementa as normas ABNT para documentos científicos, 
especialmente a NBR 6022:2018, que trata da apresentação de artigos científicos. 
O template inclui exemplos de citações, figuras, tabelas, equações e outros 
elementos comumente utilizados em artigos acadêmicos.

\vspace{\onelineskip}
 
\noindent
\textbf{Palavras-chave}: Artigo científico. ABNT. Metodologia científica. 
LaTeX. AbnTeX2.
\end{resumoumacoluna}

% ============================================================================
% ABSTRACT EM INGLÊS
% ============================================================================
\renewcommand{\resumoname}{Abstract}
\begin{resumoumacoluna}
\begin{otherlanguage*}{english}
This paper presents an overview of the preparation of scientific articles 
following the standards of the Brazilian Association of Technical Standards (ABNT). 
The abstract should contain between 150 and 250 words, presenting concisely 
the objective, methodology, main results and conclusions of the work. It should 
not contain bibliographic citations. The text should be written in a single 
paragraph, with simple spacing and justified. It is recommended to use the third 
person singular and verbs in the active voice. This template was developed using 
the abnTeX2 class, which implements ABNT standards for scientific documents, 
especially NBR 6022:2018, which deals with the presentation of scientific articles. 
The template includes examples of citations, figures, tables, equations and other 
elements commonly used in academic articles.

\vspace{\onelineskip}

\noindent
\textbf{Keywords}: Scientific article. ABNT. Scientific methodology. LaTeX. 
AbnTeX2.
\end{otherlanguage*}
\end{resumoumacoluna}

% ============================================================================
% TEXTO DO ARTIGO
% ============================================================================
\textual

% ============================================================================
\section{Introdução}
% ============================================================================
\label{sec:introducao}

A comunicação científica é fundamental para o avanço do conhecimento e a 
disseminação de novas descobertas. Os artigos científicos constituem uma das 
principais formas de divulgação de resultados de pesquisas acadêmicas 
\cite{severino2007metodologia}.

Este documento apresenta um template completo para elaboração de artigos 
científicos no formato ABNT, utilizando o sistema \LaTeX{} e a classe 
\texttt{abnTeX2}. A Associação Brasileira de Normas Técnicas estabelece 
diretrizes específicas para a formatação de trabalhos acadêmicos através 
da norma NBR 6022:2018 \cite{abnt2018}.

\subsection{Contextualização}
\label{subsec:contextualizacao}

O desenvolvimento de pesquisas científicas exige não apenas rigor metodológico, 
mas também a adequada apresentação dos resultados. A padronização dos documentos 
científicos facilita a leitura, compreensão e avaliação dos trabalhos pela 
comunidade acadêmica.

\lipsum[1]

\subsection{Justificativa}
\label{subsec:justificativa}

A necessidade de seguir normas técnicas na elaboração de documentos acadêmicos 
justifica-se pela busca de uniformização e qualidade na apresentação de trabalhos 
científicos. Como afirma \citeonline{gil2008metodos}, "a apresentação formal 
de um trabalho científico é tão importante quanto seu conteúdo".

\subsection{Objetivos}
\label{subsec:objetivos}

O objetivo deste artigo é apresentar um modelo completo e funcional para 
elaboração de artigos científicos conforme as normas ABNT. 

Os objetivos específicos incluem:

\begin{itemize}
    \item Demonstrar a estrutura básica de um artigo científico;
    \item Exemplificar o uso de citações diretas e indiretas;
    \item Ilustrar a inserção de figuras, tabelas e equações;
    \item Apresentar boas práticas na redação científica.
\end{itemize}

\subsection{Organização do Artigo}
\label{subsec:organizacao}

Este artigo está organizado da seguinte forma: a \autoref{sec:fundamentacao} 
apresenta a fundamentação teórica; a \autoref{sec:metodologia} descreve a 
metodologia utilizada; a \autoref{sec:resultados} apresenta e discute os 
resultados; e a \autoref{sec:conclusao} apresenta as conclusões e 
recomendações para trabalhos futuros.

% ============================================================================
\section{Fundamentação Teórica}
% ============================================================================
\label{sec:fundamentacao}

Esta seção apresenta os conceitos fundamentais necessários para a compreensão 
do trabalho desenvolvido.

\subsection{Normas ABNT para Artigos Científicos}
\label{subsec:normas-abnt}

A NBR 6022:2018 estabelece os princípios gerais para a apresentação de artigos 
em publicações periódicas científicas impressas. Segundo a norma, um artigo 
científico é definido como parte de uma publicação com autoria declarada, que 
apresenta e discute ideias, métodos, técnicas, processos e resultados nas 
diversas áreas do conhecimento \cite{abnt2018}.

\lipsum[2]

\subsection{Elementos de um Artigo Científico}
\label{subsec:elementos}

Um artigo científico completo deve conter elementos pré-textuais, textuais e 
pós-textuais, conforme descrito a seguir.

\subsubsection{Elementos Pré-textuais}
\label{subsubsec:pre-textuais}

Os elementos pré-textuais incluem:

\begin{enumerate}
    \item Título e subtítulo (se houver);
    \item Nome(s) do(s) autor(es);
    \item Resumo na língua do texto;
    \item Palavras-chave na língua do texto;
    \item Resumo em língua estrangeira (Abstract);
    \item Palavras-chave em língua estrangeira (Keywords).
\end{enumerate}

\subsubsection{Elementos Textuais}
\label{subsubsec:textuais}

O texto do artigo deve ser organizado em seções que apresentem:

\begin{itemize}
    \item \textbf{Introdução:} contextualização, justificativa e objetivos;
    \item \textbf{Desenvolvimento:} fundamentação teórica, metodologia e 
    apresentação dos dados/resultados;
    \item \textbf{Conclusão:} síntese dos resultados e recomendações.
\end{itemize}

\subsection{Citações e Referências}
\label{subsec:citacoes}

As citações podem ser diretas ou indiretas. Citações diretas com até três 
linhas devem vir entre aspas, como exemplo: segundo \citeonline{lakatos2003}, 
"a pesquisa científica requer criatividade, disciplina, organização e modéstia".

Citações com mais de três linhas devem ser destacadas com recuo de 4 cm, fonte 
menor e sem aspas:

\begin{citacao}
A pesquisa científica não é apenas um procedimento formal, com método de 
pensamento reflexivo, que requer um tratamento científico e se constitui no 
caminho para conhecer a realidade ou para descobrir verdades parciais. Significa 
muito mais do que apenas procurar a verdade: é encontrar respostas para questões 
propostas, utilizando métodos científicos \cite{lakatos2003}.
\end{citacao}

Citações indiretas (paráfrase) não necessitam de aspas, apenas a indicação da 
fonte, como: a metodologia científica envolve procedimentos sistemáticos para 
obtenção de conhecimento \cite{gil2008metodos}.

% ============================================================================
\section{Metodologia}
% ============================================================================
\label{sec:metodologia}

Esta seção descreve os procedimentos metodológicos adotados neste trabalho.

\subsection{Tipo de Pesquisa}
\label{subsec:tipo-pesquisa}

Quanto à natureza, esta é uma pesquisa aplicada. Quanto aos objetivos, 
caracteriza-se como descritiva. Quanto aos procedimentos técnicos, trata-se 
de uma pesquisa bibliográfica \cite{gil2008metodos}.

\lipsum[3]

\subsection{Instrumentos de Coleta de Dados}
\label{subsec:instrumentos}

\lipsum[4]

\subsection{Procedimentos de Análise}
\label{subsec:procedimentos}

\lipsum[5]

% ============================================================================
\section{Resultados e Discussão}
% ============================================================================
\label{sec:resultados}

Esta seção apresenta os resultados obtidos e a discussão sobre os mesmos.

\subsection{Exemplo de Figura}
\label{subsec:figura}

As figuras devem ser citadas no texto e centralizadas na página. A 
\autoref{fig:exemplo} ilustra um exemplo de inserção de figura.

\begin{figure}[htb]
    \centering
    \caption{Exemplo de inserção de figura em artigo científico}
    \label{fig:exemplo}
    % Para inserir uma imagem real, descomente a linha abaixo:
    % \includegraphics[width=0.6\textwidth]{imagens/exemplo.png}
    % Por enquanto, usamos um placeholder:
    \fbox{\parbox{0.6\textwidth}{\centering\vspace{3cm}
    Insira sua figura aqui\\
    (use o comando includegraphics)
    \vspace{3cm}}}
    \legend{Fonte: Elaborado pelos autores (2024).}
\end{figure}

Observe que a legenda da figura fica acima da imagem, enquanto a fonte fica 
abaixo, conforme normas ABNT.

\subsection{Exemplo de Tabela}
\label{subsec:tabela}

As tabelas devem seguir o padrão IBGE, sem linhas verticais. A 
\autoref{tab:exemplo} apresenta um exemplo de tabela.

\begin{table}[htb]
    \centering
    \caption{Resultados da análise comparativa de métodos}
    \label{tab:exemplo}
    \begin{tabular}{lccc}
        \toprule
        \textbf{Método} & \textbf{Precisão (\%)} & \textbf{Recall (\%)} & \textbf{F1-Score} \\
        \midrule
        Método A & 87.5 & 82.3 & 0.848 \\
        Método B & 91.2 & 88.7 & 0.899 \\
        Método C & 89.8 & 85.1 & 0.874 \\
        Proposta & \textbf{93.4} & \textbf{90.2} & \textbf{0.917} \\
        \bottomrule
    \end{tabular}
    \legend{Fonte: Dados da pesquisa (2024).}
\end{table}

Note que a legenda da tabela fica acima e a fonte fica abaixo, seguindo o 
padrão ABNT.

\subsection{Exemplo de Equação}
\label{subsec:equacao}

Equações devem ser numeradas sequencialmente. A \autoref{eq:pitagoras} 
apresenta o teorema de Pitágoras.

\begin{equation}
    a^2 + b^2 = c^2
    \label{eq:pitagoras}
\end{equation}

\noindent onde $a$ e $b$ são os catetos e $c$ é a hipotenusa de um triângulo 
retângulo.

Para equações mais complexas, pode-se usar o ambiente \texttt{align}:

\begin{align}
    f(x) &= \int_{-\infty}^{\infty} e^{-x^2} dx \label{eq:integral} \\
    &= \sqrt{\pi} \label{eq:resultado}
\end{align}

\subsection{Exemplo de Lista}
\label{subsec:lista}

As listas podem ser numeradas ou não numeradas. Exemplo de lista numerada:

\begin{enumerate}
    \item Primeiro item da lista numerada;
    \item Segundo item da lista numerada;
    \item Terceiro item da lista numerada.
\end{enumerate}

Exemplo de lista não numerada:

\begin{itemize}
    \item Primeiro item da lista não numerada;
    \item Segundo item da lista não numerada;
    \item Terceiro item da lista não numerada.
\end{itemize}

\subsection{Discussão dos Resultados}
\label{subsec:discussao}

\lipsum[6-7]

% ============================================================================
\section{Conclusão}
% ============================================================================
\label{sec:conclusao}

Este artigo apresentou um template completo para elaboração de artigos 
científicos seguindo as normas ABNT, utilizando o sistema \LaTeX{} e a 
classe \texttt{abnTeX2}.

\subsection{Principais Contribuições}
\label{subsec:contribuicoes}

As principais contribuições deste trabalho incluem:

\begin{enumerate}
    \item Apresentação de um template funcional e documentado;
    \item Exemplificação de todos os elementos principais de um artigo;
    \item Demonstração de boas práticas em formatação acadêmica;
    \item Facilitação do processo de redação científica.
\end{enumerate}

\subsection{Limitações do Estudo}
\label{subsec:limitacoes}

\lipsum[8]

\subsection{Recomendações para Trabalhos Futuros}
\label{subsec:trabalhos-futuros}

Como trabalhos futuros, sugere-se:

\begin{itemize}
    \item Desenvolvimento de templates específicos para diferentes áreas;
    \item Criação de versões em duas colunas para periódicos específicos;
    \item Integração com ferramentas de gerenciamento de referências;
    \item Desenvolvimento de templates para artigos em inglês.
\end{itemize}

% ============================================================================
% ELEMENTOS PÓS-TEXTUAIS
% ============================================================================
\postextual

% ============================================================================
% REFERÊNCIAS BIBLIOGRÁFICAS
% ============================================================================
\bibliography{referencias}

% ============================================================================
% GLOSSÁRIO (Opcional)
% ============================================================================
% \glossary

% ============================================================================
% APÊNDICES (Opcional)
% ============================================================================
% \begin{apendicesenv}
% \partapendices
% 
% \section{Primeiro Apêndice}
% \label{apend:A}
% 
% Conteúdo do primeiro apêndice.
% 
% \end{apendicesenv}

% ============================================================================
% ANEXOS (Opcional)
% ============================================================================
% \begin{anexosenv}
% \partanexos
% 
% \section{Primeiro Anexo}
% \label{anex:A}
% 
% Conteúdo do primeiro anexo.
% 
% \end{anexosenv}

% ============================================================================
% AGRADECIMENTOS (Opcional - pode vir antes das referências)
% ============================================================================
\section*{Agradecimentos}

Os autores agradecem à Universidade Federal de Exemplo pelo apoio institucional 
e à agência de fomento XYZ pelo apoio financeiro (Processo nº 123456/2024-0).

\end{document}

% ============================================================================
% ARQUIVO referencias.bib (criar separadamente)
% ============================================================================
% Exemplo de entradas bibliográficas no formato ABNT:
%
% @book{severino2007metodologia,
%   author = {Severino, Antônio Joaquim},
%   title = {Metodologia do trabalho científico},
%   publisher = {Cortez},
%   address = {São Paulo},
%   edition = {23},
%   year = {2007}
% }
%
% @manual{abnt2018,
%   organization = {ABNT},
%   title = {NBR 6022: Informação e documentação - Artigo em publicação 
%            periódica técnica e/ou científica - Apresentação},
%   address = {Rio de Janeiro},
%   year = {2018}
% }
%
% @book{lakatos2003,
%   author = {Lakatos, Eva Maria and Marconi, Marina de Andrade},
%   title = {Fundamentos de metodologia científica},
%   publisher = {Atlas},
%   address = {São Paulo},
%   edition = {5},
%   year = {2003}
% }
%
% @book{gil2008metodos,
%   author = {Gil, Antônio Carlos},
%   title = {Métodos e técnicas de pesquisa social},
%   publisher = {Atlas},
%   address = {São Paulo},
%   edition = {6},
%   year = {2008}
% }
% ============================================================================