% ============================================================================
% TEMPLATE PARA TESE DE DOUTORADO / DISSERTAÇÃO DE MESTRADO
% Baseado em abnTeX2 - Normas ABNT NBR 14724:2011
% ============================================================================

\documentclass[
    % Opções da classe memoir
    12pt,               % Tamanho da fonte
    openright,          % Capítulos começam em página ímpar (insere página vazia se necessário)
    twoside,            % Para impressão frente e verso. Use oneside para digital
    a4paper,            % Tamanho do papel
    % Opções da classe abntex2
    %chapter=TITLE,     % Títulos de capítulos em maiúsculas
    %section=TITLE,     % Títulos de seções em maiúsculas
    %subsection=TITLE,  % Títulos de subseções em maiúsculas
    %subsubsection=TITLE,% Títulos de subsubseções em maiúsculas
    % Opções do pacote babel
    english,            % Idioma adicional para hifenização
    french,             % Idioma adicional para hifenização
    spanish,            % Idioma adicional para hifenização
    brazil              % O último idioma é o principal do documento
]{abntex2}

% ==================== PACOTES ESSENCIAIS ====================================
\usepackage{lmodern}            % Usa a fonte Latin Modern
\usepackage[T1]{fontenc}        % Seleção de códigos de fonte
\usepackage[utf8]{inputenc}     % Codificação do documento (conversão automática dos acentos)
\usepackage{indentfirst}        % Indenta o primeiro parágrafo de cada seção
\usepackage{color}              % Controle das cores
\usepackage{graphicx}           % Inclusão de gráficos
\usepackage{microtype}          % Para melhorias de justificação
\usepackage{lipsum}             % Geração de texto dummy (remover em versão final)

% ==================== PACOTES DE CITAÇÕES ===================================
\usepackage[brazilian,hyperpageref]{backref}    % Páginas com citações na bibliografia
\usepackage[alf]{abntex2cite}   % Citações padrão ABNT

% ==================== PACOTES ADICIONAIS ====================================
\usepackage{amsmath,amssymb,amsfonts}  % Matemática avançada
\usepackage{algorithm}                  % Algoritmos
\usepackage{algpseudocode}             % Pseudocódigo
\usepackage{listings}                   % Código fonte
\usepackage{multirow}                   % Tabelas com múltiplas linhas
\usepackage{booktabs}                   % Tabelas profissionais
\usepackage{rotating}                   % Rotação de objetos
\usepackage{subfig}                     % Subfiguras
\usepackage{pdfpages}                   % Inclusão de PDFs

% ==================== CONFIGURAÇÕES DE CORES ================================
\definecolor{blue}{RGB}{41,5,195}

% ==================== CONFIGURAÇÕES DE LISTAGENS ============================
\lstset{
    basicstyle=\footnotesize\ttfamily,
    numbers=left,
    numberstyle=\tiny,
    stepnumber=1,
    numbersep=5pt,
    backgroundcolor=\color{white},
    showspaces=false,
    showstringspaces=false,
    showtabs=false,
    frame=single,
    rulecolor=\color{black},
    tabsize=2,
    captionpos=b,
    breaklines=true,
    breakatwhitespace=false,
    title=\lstname,
}

% ==================== INFORMAÇÕES DO DOCUMENTO ==============================
\titulo{Título do Trabalho: Subtítulo se Houver}
\autor{Nome Completo do Autor}
\local{Nome da Cidade}
\data{2024}
\orientador{Prof. Dr. Nome do Orientador}
\coorientador{Prof. Dr. Nome do Coorientador} % Comentar se não houver

\instituicao{%
    Universidade Federal de Exemplo -- UFE
    \par
    Instituto de Ciências Exatas
    \par
    Programa de Pós-Graduação em Ciência da Computação}

\tipotrabalho{Tese (Doutorado)} % Ou: Dissertação (Mestrado)

% O preâmbulo deve conter o tipo do trabalho, o objetivo, o nome da instituição
% e a área de concentração
\preambulo{Tese apresentada ao Programa de Pós-Graduação em Ciência da Computação 
da Universidade Federal de Exemplo como requisito parcial para obtenção do título 
de Doutor em Ciência da Computação.}
% Para dissertação: Dissertação apresentada... título de Mestre...

% ==================== CONFIGURAÇÕES DE APARÊNCIA ============================
% Alterando o aspecto da cor azul
\definecolor{blue}{RGB}{41,5,195}

% Informações do PDF
\makeatletter
\hypersetup{
    pdftitle={\@title}, 
    pdfauthor={\@author},
    pdfsubject={\imprimirpreambulo},
    pdfcreator={LaTeX with abnTeX2},
    pdfkeywords={palavra-chave1}{palavra-chave2}{palavra-chave3}, 
    colorlinks=true,        % false: boxed links; true: colored links
    linkcolor=blue,         % color of internal links
    citecolor=blue,         % color of links to bibliography
    filecolor=magenta,      % color of file links
    urlcolor=blue,
    bookmarksdepth=4
}
\makeatother

% Espaçamentos entre linhas e parágrafos
\setlength{\parindent}{1.3cm}
\setlength{\parskip}{0.2cm}

% Compila o índice
\makeindex

% ============================================================================
% INÍCIO DO DOCUMENTO
% ============================================================================
\begin{document}

% Seleciona o idioma do documento (conforme pacotes do babel)
\selectlanguage{brazil}

% Retira espaço extra obsoleto entre as frases.
\frenchspacing 

% ============================================================================
% ELEMENTOS PRÉ-TEXTUAIS
% ============================================================================
\pretextual

% ---
% Capa
% ---
\imprimircapa

% ---
% Folha de rosto
% (o * indica que haverá a ficha bibliográfica)
% ---
\imprimirfolhaderosto*

% ---
% Inserir a ficha bibliográfica
% ---
% Isto é um exemplo de Ficha Catalográfica, ou "Dados internacionais de
% catalogação-na-publicação". Você pode utilizar este modelo como referência. 
% Porém, provavelmente a biblioteca da sua universidade lhe fornecerá um PDF
% com a ficha catalográfica definitiva após a defesa do trabalho. Quando estiver
% com o documento, salve-o como PDF no diretório do seu projeto e substitua todo
% o conteúdo de implementação deste arquivo pelo comando abaixo:
%
% \begin{fichacatalografica}
%     \includepdf{fig_ficha_catalografica.pdf}
% \end{fichacatalografica}

\begin{fichacatalografica}
    \sffamily
    \vspace*{\fill}                 % Posição vertical
    \begin{center}                  % Minipage Centralizado
    \fbox{\begin{minipage}[c][8cm]{13.5cm}       % Largura
    \small
    \imprimirautor
    %Sobrenome, Nome do autor
    
    \hspace{0.5cm} \imprimirtitulo  / \imprimirautor. --
    \imprimirlocal, \imprimirdata-
    
    \hspace{0.5cm} \pageref{LastPage} p. : il. (algumas color.) ; 30 cm.\\
    
    \hspace{0.5cm} \imprimirorientadorRotulo~\imprimirorientador\\
    
    \hspace{0.5cm}
    \parbox[t]{\textwidth}{\imprimirtipotrabalho~--~\imprimirinstituicao,
    \imprimirdata.}\\
    
    \hspace{0.5cm}
        1. Palavra-chave1.
        2. Palavra-chave2.
        3. Palavra-chave3.
        I. Orientador.
        II. Universidade xxx.
        III. Faculdade de xxx.
        IV. Título\\ 
    
    \hspace{8.75cm} CDU 02:141:005.7\\
    
    \end{minipage}}
    \end{center}
\end{fichacatalografica}

% ---
% Inserir folha de aprovação
% ---
% Isto é um exemplo de Folha de aprovação, elemento obrigatório da NBR
% 14724/2011 (seção 4.2.1.3). Você pode utilizar este modelo até a aprovação
% do trabalho. Após isso, substitua todo o conteúdo deste arquivo por uma
% imagem da página assinada pela banca com o comando abaixo:
%
% \begin{folhadeaprovacao}
% \includepdf{folhadeaprovacao_final.pdf}
% \end{folhadeaprovacao}
%
\begin{folhadeaprovacao}

  \begin{center}
    {\ABNTEXchapterfont\large\imprimirautor}

    \vspace*{\fill}\vspace*{\fill}
    \begin{center}
      \ABNTEXchapterfont\bfseries\Large\imprimirtitulo
    \end{center}
    \vspace*{\fill}
    
    \hspace{.45\textwidth}
    \begin{minipage}{.5\textwidth}
        \imprimirpreambulo
    \end{minipage}%
    \vspace*{\fill}
   \end{center}
        
   Trabalho aprovado. \imprimirlocal, \underline{\hspace{2cm}} de \underline{\hspace{4cm}} de \imprimirdata:

   \assinatura{\textbf{\imprimirorientador} \\ Orientador} 
   \assinatura{\textbf{Prof. Dr. Nome Completo} \\ Instituição}
   \assinatura{\textbf{Prof. Dr. Nome Completo} \\ Instituição}
   \assinatura{\textbf{Prof. Dr. Nome Completo} \\ Instituição}
   %\assinatura{\textbf{Professor} \\ Convidado 4}
      
   \begin{center}
    \vspace*{0.5cm}
    {\large\imprimirlocal}
    \par
    {\large\imprimirdata}
    \vspace*{1cm}
  \end{center}
  
\end{folhadeaprovacao}

% ---
% Dedicatória (Opcional)
% ---
\begin{dedicatoria}
   \vspace*{\fill}
   \centering
   \noindent
   \textit{Este trabalho é dedicado aos meus pais e família...} \vspace*{\fill}
\end{dedicatoria}

% ---
% Agradecimentos (Opcional)
% ---
\begin{agradecimentos}
Agradeço primeiramente a Deus...

Aos meus orientadores...

À minha família...

Aos colegas do laboratório...

Às agências de fomento...

\end{agradecimentos}

% ---
% Epígrafe (Opcional)
% ---
\begin{epigrafe}
    \vspace*{\fill}
    \begin{flushright}
        \textit{``Não vos amoldeis às estruturas deste mundo, \\
        mas transformai-vos pela renovação da mente, \\
        a fim de distinguir qual é a vontade de Deus: \\
        o que é bom, o que Lhe é agradável, o que é perfeito.''\\
        (Bíblia Sagrada, Romanos 12, 2)}
    \end{flushright}
\end{epigrafe}

% ---
% RESUMOS
% ---

% Resumo em português
\setlength{\absparsep}{18pt} % ajusta o espaçamento dos parágrafos do resumo
\begin{resumo}
Este é o resumo do trabalho. Deve conter entre 150 e 500 palavras. 
Deve apresentar de forma concisa os objetivos, a metodologia, os resultados 
e as conclusões do trabalho. Não deve conter citações. Evite usar siglas.
\lipsum[1]

 \textbf{Palavras-chave}: Palavra-chave 1. Palavra-chave 2. Palavra-chave 3.
\end{resumo}

% Resumo em inglês
\begin{resumo}[Abstract]
 \begin{otherlanguage*}{english}
   This is the english abstract. It should contain between 150 and 500 words.
   It must concisely present the objectives, methodology, results and 
   conclusions of the work. It should not contain citations. Avoid using acronyms.
   \lipsum[1]

   \vspace{\onelineskip}
 
   \noindent 
   \textbf{Keywords}: Keyword 1. Keyword 2. Keyword 3.
 \end{otherlanguage*}
\end{resumo}

% ---
% LISTAS
% ---

% Lista de ilustrações
\pdfbookmark[0]{\listfigurename}{lof}
\listoffigures*
\cleardoublepage

% Lista de tabelas
\pdfbookmark[0]{\listtablename}{lot}
\listoftables*
\cleardoublepage

% Lista de abreviaturas e siglas
\begin{siglas}
  \item[ABNT] Associação Brasileira de Normas Técnicas
  \item[API] Application Programming Interface
  \item[CNN] Convolutional Neural Network
  \item[GPU] Graphics Processing Unit
  \item[HTTP] HyperText Transfer Protocol
  \item[IA] Inteligência Artificial
  \item[ML] Machine Learning
  \item[RNA] Rede Neural Artificial
  \item[TCC] Trabalho de Conclusão de Curso
  \item[UFE] Universidade Federal de Exemplo
\end{siglas}

% Lista de símbolos
\begin{simbolos}
  \item[$\alpha$] Alfa
  \item[$\beta$] Beta
  \item[$\gamma$] Gama
  \item[$\delta$] Delta
  \item[$\epsilon$] Epsilon
  \item[$\theta$] Theta
  \item[$\lambda$] Lambda
  \item[$\mu$] Mi
  \item[$\pi$] Pi
  \item[$\sigma$] Sigma
  \item[$\Omega$] Ômega
  \item[$\in$] Pertence
  \item[$\forall$] Para todo
  \item[$\exists$] Existe
  \item[$\sum$] Somatório
  \item[$\int$] Integral
\end{simbolos}

% ---
% Inserir o sumário
% ---
\pdfbookmark[0]{\contentsname}{toc}
\tableofcontents*
\cleardoublepage

% ============================================================================
% ELEMENTOS TEXTUAIS
% ============================================================================
\textual

% ============================================================================
% INTRODUÇÃO
% ============================================================================
\chapter{Introdução}
\label{cap:introducao}

Este documento é um exemplo de template para elaboração de teses de doutorado 
e dissertações de mestrado utilizando a classe \texttt{abntex2}, que implementa 
as normas ABNT, especialmente a NBR 14724:2011.

\section{Contextualização}

Apresente aqui o contexto geral do trabalho, situando o leitor sobre a área 
de pesquisa e sua importância. \lipsum[2]

\section{Motivação}

Explique o que motivou a realização desta pesquisa, quais problemas ou lacunas 
você identificou na literatura. \lipsum[3]

\section{Objetivos}

\subsection{Objetivo Geral}

O objetivo geral deste trabalho é... \lipsum[4]

\subsection{Objetivos Específicos}

Os objetivos específicos são:

\begin{itemize}
    \item Objetivo específico 1...
    \item Objetivo específico 2...
    \item Objetivo específico 3...
    \item Objetivo específico 4...
\end{itemize}

\section{Contribuições}

Este trabalho traz as seguintes contribuições para a área:

\begin{enumerate}
    \item Contribuição 1...
    \item Contribuição 2...
    \item Contribuição 3...
\end{enumerate}

\section{Organização do Trabalho}

Este trabalho está organizado da seguinte forma:

\begin{itemize}
    \item O \autoref{cap:fundamentacao} apresenta a fundamentação teórica...
    \item O \autoref{cap:trabalhos} discute os trabalhos relacionados...
    \item O \autoref{cap:proposta} descreve a proposta desenvolvida...
    \item O \autoref{cap:resultados} apresenta os resultados obtidos...
    \item O \autoref{cap:conclusao} conclui o trabalho e aponta direções futuras...
\end{itemize}

% ============================================================================
% FUNDAMENTAÇÃO TEÓRICA
% ============================================================================
\chapter{Fundamentação Teórica}
\label{cap:fundamentacao}

Este capítulo apresenta os conceitos fundamentais necessários para a compreensão 
do trabalho desenvolvido.

\section{Conceito Fundamental 1}

Explique o primeiro conceito fundamental. Use citações como \citeonline{knuth1984} 
ou \cite{lamport1994}.

\lipsum[5-6]

\subsection{Subconceito 1.1}

\lipsum[7]

\subsection{Subconceito 1.2}

\lipsum[8]

\section{Conceito Fundamental 2}

\lipsum[9-10]

\section{Exemplos de Elementos}

\subsection{Exemplo de Figura}

A \autoref{fig:exemplo} mostra um exemplo de como inserir figuras.

\begin{figure}[htb]
    \centering
    \caption{Exemplo de figura}
    % \includegraphics[width=0.5\textwidth]{imagens/exemplo.png}
    \fbox{Insira sua figura aqui}
    \label{fig:exemplo}
    \legend{Fonte: Elaborado pelo autor (2024)}
\end{figure}

\subsection{Exemplo de Tabela}

A \autoref{tab:exemplo} apresenta um exemplo de tabela.

\begin{table}[htb]
    \centering
    \caption{Exemplo de tabela}
    \label{tab:exemplo}
    \begin{tabular}{lcc}
        \toprule
        \textbf{Item} & \textbf{Valor 1} & \textbf{Valor 2} \\
        \midrule
        Item A & 10.5 & 20.3 \\
        Item B & 15.2 & 18.7 \\
        Item C & 12.8 & 22.1 \\
        \bottomrule
    \end{tabular}
    \legend{Fonte: Elaborado pelo autor (2024)}
\end{table}

\subsection{Exemplo de Equação}

A equação de uma reta pode ser expressa como mostrado na \autoref{eq:reta}.

\begin{equation}
    y = ax + b
    \label{eq:reta}
\end{equation}

Onde $y$ é a variável dependente, $x$ é a variável independente, $a$ é o 
coeficiente angular e $b$ é o coeficiente linear.

\subsection{Exemplo de Citação Direta Longa}

Para citações com mais de três linhas, use o ambiente \texttt{citacao}:

\begin{citacao}
Esta é uma citação direta longa, com mais de três linhas. Ela deve ter recuo 
de 4 cm da margem esquerda, texto em fonte menor (tamanho 10), sem aspas e com 
espaçamento simples entre linhas. Este formato segue as normas da ABNT para 
citações longas em trabalhos acadêmicos \cite{abnt2011}.
\end{citacao}

% ============================================================================
% TRABALHOS RELACIONADOS
% ============================================================================
\chapter{Trabalhos Relacionados}
\label{cap:trabalhos}

Este capítulo apresenta e discute os principais trabalhos relacionados à 
pesquisa desenvolvida.

\section{Trabalho Relacionado 1}

\lipsum[11-12]

\section{Trabalho Relacionado 2}

\lipsum[13-14]

\section{Análise Comparativa}

\lipsum[15]

% ============================================================================
% PROPOSTA / METODOLOGIA
% ============================================================================
\chapter{Proposta}
\label{cap:proposta}

Este capítulo descreve detalhadamente a solução/metodologia proposta neste trabalho.

\section{Visão Geral}

\lipsum[16-17]

\section{Arquitetura}

\lipsum[18-19]

\section{Detalhamento da Implementação}

\lipsum[20-21]

% ============================================================================
% RESULTADOS E DISCUSSÃO
% ============================================================================
\chapter{Resultados e Discussão}
\label{cap:resultados}

Este capítulo apresenta os resultados obtidos e a discussão sobre os mesmos.

\section{Configuração dos Experimentos}

\lipsum[22]

\section{Resultados Obtidos}

\lipsum[23-24]

\section{Discussão}

\lipsum[25-26]

% ============================================================================
% CONCLUSÃO
% ============================================================================
\chapter{Conclusão}
\label{cap:conclusao}

Este capítulo apresenta as conclusões do trabalho e as perspectivas para 
trabalhos futuros.

\section{Considerações Finais}

\lipsum[27-28]

\section{Contribuições}

As principais contribuições deste trabalho são:

\begin{enumerate}
    \item Contribuição 1 revisitada...
    \item Contribuição 2 revisitada...
    \item Contribuição 3 revisitada...
\end{enumerate}

\section{Limitações}

\lipsum[29]

\section{Trabalhos Futuros}

Como trabalhos futuros, sugere-se:

\begin{itemize}
    \item Extensão 1...
    \item Extensão 2...
    \item Extensão 3...
\end{itemize}

% ============================================================================
% ELEMENTOS PÓS-TEXTUAIS
% ============================================================================
\postextual

% ============================================================================
% REFERÊNCIAS BIBLIOGRÁFICAS
% ============================================================================
\bibliography{referencias}

% ============================================================================
% GLOSSÁRIO (Opcional)
% ============================================================================
%\glossary

% ============================================================================
% APÊNDICES (Opcional)
% ============================================================================
\begin{apendicesenv}

\partapendices

\chapter{Primeiro Apêndice}
\label{apend:primeiro}

Apêndices são textos ou documentos elaborados pelo próprio autor.

\lipsum[30-31]

\chapter{Segundo Apêndice}
\label{apend:segundo}

\lipsum[32-33]

\end{apendicesenv}

% ============================================================================
% ANEXOS (Opcional)
% ============================================================================
\begin{anexosenv}

\partanexos

\chapter{Primeiro Anexo}
\label{anex:primeiro}

Anexos são textos ou documentos não elaborados pelo autor.

\lipsum[34-35]

\chapter{Segundo Anexo}
\label{anex:segundo}

\lipsum[36-37]

\end{anexosenv}

% ============================================================================
% ÍNDICE REMISSIVO (Opcional)
% ============================================================================
%\phantompart
%\printindex

\end{document}

% ============================================================================
% ARQUIVO referencias.bib (criar separadamente)
% ============================================================================
% Exemplo de entradas bibliográficas:
%
% @book{knuth1984,
%   author = {Donald E. Knuth},
%   title = {The TeXbook},
%   publisher = {Addison-Wesley},
%   year = {1984}
% }
%
% @book{lamport1994,
%   author = {Leslie Lamport},
%   title = {LaTeX: A Document Preparation System},
%   publisher = {Addison-Wesley},
%   edition = {2nd},
%   year = {1994}
% }
%
% @manual{abnt2011,
%   organization = {ABNT},
%   title = {NBR 14724: Informação e documentação - Trabalhos acadêmicos - Apresentação},
%   address = {Rio de Janeiro},
%   year = {2011}
% }
% ============================================================================